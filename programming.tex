\chapter{Programming}
\section{GIT}
	\begin{questions}{Setup}
		\begin{questionAnswer}
			\qItem{Get a repo}{git clone}
			\qItem{Make a repo}{git init}
			\qItem{Pull an existing repo}{Use init or clone the repo then pull}
			\qItem{Remote repos}{git remote \ra lists the remote repos, git remote add \opt{name} \opt{url}. \opt{name} is usually origin.}
			\qItem{Configuration}{git config \ra complicated, but add email and user with git config $--$global user.email \& user.name}
		\end{questionAnswer}
	\end{questions}

\section{Terms}
\begin{questions}{Programming}
	\begin{enumerate}
		\label{sect:agile}
		\item \textbf{Agile: } Software development strategy. Values:
			\begin{enumerate}
				\item \textbf{Individuals and Interactions} over processes and tools
					\begin{enumerate}
						\item Pair programming \ra 1 station 2 programmers, driver \& navigator/observer
						\item Colocation \ra Team members in the same area
					\end{enumerate}
				\item \textbf{Working Software} over comprehensive documentation
				\item \textbf{Customer Collaboration} over contract negotiation
				\item \textbf{Responding to Change} over following a plan
			\end{enumerate}
	\end{enumerate}
\end{questions}

%\section{C/C++}

\section{Python}
	\begin{questions}{Standard Library Scripting}
		\rsrc{Python Standard Library}\url{https://docs.python.org/3/library/index.html}
		\begin{questionAnswer}
			\qItem{OS Module}{Miscellaneous operating system interfaces. Some attributes are cross platform, some platform specific. OS.path contains all the file name manipulation tools.}
			\qItem{Subprocess Module}{Meant to replace parts of the OS module. Run subprocesses, use pipes, etc.}
			\qItem{Sys Module}{Use for passing simple arguments (use argparse for more complicated argument passing). Get system information and shell variable analogs for the python environment.}
			\qItem{Argparse}{A more plush way of getting command line arguments. Auto-generates a help screen.}
			\qItem{Shutil Module}{Contains file and directory functions}
		\end{questionAnswer}
	\end{questions}

	\begin{questions}{Package Development}
		\rsrc{Pydocs} \url{https://docs.python.org/2/tutorial/modules.html#packages}
		\rsrc{PCU} \url{https://pythonconquerstheuniverse.wordpress.com/2009/10/15/python-packages/}
		\begin{questionAnswer}
			\qItem{Making a Package}{\begin{enumerate}
					\item Contains a set of modules and atleast one \_\_init\_\_.py
					\item Append the location of the module to PYTHONPATH, the working directory is checked last \ra sys.path.append(\opt{Package Location})
					\item The \_\_init\_\_.py module is run at the start, so to have all submodules nicely loaded use from add import add to be able to call \opt{Package Name}.add instead of \opt{Package Name}.add.add
				\end{enumerate}}
		\end{questionAnswer}
	\end{questions}

	\begin{questions}{Environment}
		\begin{questionAnswer}
			\qItem{Virtualenv}{Isolated working copy of Python allowing the altering of a python setup without affecting other projects}
			\qItem{Use}{Packages installed here will not affect the global Python installation.}
			\qItem{New virtualenv}{In a clean directory run: virtualenv \opt{Dir Name}, add $--$no-site-packages to not use already installed packages}
			\qItem{Add packages}{Call pip from the correct env directory. This will install the package in the virtual environment directory instead of the main installation}
			\qItem{Activate}{From \opt{Dir Name} use: source activate.}
			\qItem{Deactivate}{Call: deactivate}
		\end{questionAnswer}
	\end{questions}

	\begin{questions}{Python Terms}
		\begin{questionAnswer}
			\qItem{Bind}{Assign/Set an attribute}
			\qItem{Attribute}{State/variable of an object}
			\qItem{Access}{Get a value from a variable}
			\qItem{method}{Class function}
			\qItem{Constructor/ initializer}{\_\_init\_\_ ('dunder init')}
			\qItem{Destructor}{\_\_del\_\_, if this is a class method it must be explicitly called w/ the del operator for the object to be removed from memory. del decrements the reference count for the object by one. When the reference count is zero, the destructor is called}
			\qItem{Decorator}{A wrapper function that takes a function as an argument, placed above the wrapping function w/ @\opt{decorator name}}
			\qItem{Property}{Built in decorator to assign an immutable value returned by the wrapped method}
		\end{questionAnswer}
	\end{questions}

	\begin{questions}{OOP Terms}
		\begin{questionAnswer}
			\qItem{Singleton}{Design Pattern that prevents multiple instantiation}
			\qItem{Borg}{Multiple intances share state}
			\qItem{First class objects}{Objects can be used in the same way other data types can, passed to functions assigned }
			\qItem{Super/Base/Parent}{The class/es that a derived/child/sub/child class inherits from. All 3 are synonymous}
			\qItem{derived/child/sub/child}{ The classes that inherit from a Super/Base/Parent class}
			\qItem{Overriding}{A derived class redefines a method that exists in the base class, the derived class's method is then called instead}
			\qItem{HAS-A}{}
			\qItem{IS-A}{}
		\end{questionAnswer}
	\end{questions}

	\begin{questions}{Python OOP}
		Python is polymorphic, everything-is-an-object, multi-inheriting,
	\end{questions}

	\begin{questions}{Terms}
		\begin{questionAnswer}
			\qItem{Memoization}{Saving the result of a function call so as to skip recalculation on repeated calls}
			\qItem{Polymorphic}{Single interface to different types.}
		\end{questionAnswer}
	\end{questions}

\section{MySQL}
	\begin{questions}{Relational Database}
		\rsrc{Wiki}\url{https://en.wikipedia.org/wiki/Relational_database}
		\begin{questionAnswer}
			\qItem{SQL}{Structured Query Language}
			\qItem{Relational Model}{Table rows have unique keys. This allows for columns of 1 table to be linked to columns in another on some shared attribute}
			\qItem{Databases}{\begin{enumerate}
				\item MySQL
				\item Mariadb
				\item sqllite
				\item Postgres
				\item couchdb
			\end{enumerate}}
		\end{questionAnswer}
	\end{questions}

%\section{Postgres}

\textbf{Start mysql server: } rcmysql start
