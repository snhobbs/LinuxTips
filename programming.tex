\chapter{Programming}
\section{GIT}
	\begin{questions}{Setup}
		\begin{questionAnswer}
			\qItem{Get a repo}{git clone}
			\qItem{Make a repo}{git init}
			\qItem{Pull an existing repo}{Use init or clone the repo then pull}
			\qItem{Remote repos}{git remote \ra lists the remote repos, git remote add \opt{name} \opt{url}. \opt{name} is usually origin.}
			\qItem{Configuration}{git config \ra complicated, but add email and user with git config $--$global user.email \& user.name}
		\end{questionAnswer}
	\end{questions}

\section{Terms}
\begin{questions}{Programming}
	\begin{enumerate}
		\label{sect:agile}
		\item \textbf{Agile: } Software development strategy. Values:
			\begin{enumerate}
				\item \textbf{Individuals and Interactions} over processes and tools
					\begin{enumerate}
						\item Pair programming \ra 1 station 2 programmers, driver \& navigator/observer
						\item Colocation \ra Team members in the same area
					\end{enumerate}
				\item \textbf{Working Software} over comprehensive documentation
				\item \textbf{Customer Collaboration} over contract negotiation
				\item \textbf{Responding to Change} over following a plan
			\end{enumerate}
	\end{enumerate}
\end{questions}

%\section{C/C++}

\section{Python}
	\begin{questions}{Standard Library Scripting}
		\rsrc{Python Standard Library}\url{https://docs.python.org/3/library/index.html}
		\begin{questionAnswer}
			\qItem{OS Module}{Miscellaneous operating system interfaces. Some attributes are cross platform, some platform specific. OS.path contains all the file name manipulation tools.}
			\qItem{Subprocess Module}{Meant to replace parts of the OS module. Run subprocesses, use pipes, etc.}
			\qItem{Sys Module}{Use for passing simple arguments (use argparse for more complicated argument passing). Get system information and shell variable analogs for the python environment.}
			\qItem{Argparse}{A more plush way of getting command line arguments. Auto-generates a help screen.}
			\qItem{Shutil Module}{Contains file and directory functions}
		\end{questionAnswer}
	\end{questions}

	\begin{questions}{Package Development}
		\rsrc{Pydocs} \url{https://docs.python.org/2/tutorial/modules.html#packages}
		\rsrc{PCU} \url{https://pythonconquerstheuniverse.wordpress.com/2009/10/15/python-packages/}
		\begin{questionAnswer}
			\qItem{Making a Package}{\begin{enumerate}
					\item Contains a set of modules and atleast one \_\_init\_\_.py
					\item Append the location of the module to PYTHONPATH, the working directory is checked last \ra sys.path.append(\opt{Package Location})
					\item The \_\_init\_\_.py module is run at the start, so to have all submodules nicely loaded use from add import add to be able to call \opt{Package Name}.add instead of \opt{Package Name}.add.add
				\end{enumerate}}
		\end{questionAnswer}
	\end{questions}

	\begin{questions}{Environment}
		\begin{questionAnswer}
			\qItem{Virtualenv}{Isolated working copy of Python allowing the altering of a python setup without affecting other projects}
			\qItem{Use}{Packages installed here will not affect the global Python installation.}
			\qItem{New virtualenv}{In a clean directory run: virtualenv \opt{Dir Name}, add $--$no-site-packages to not use already installed packages}
			\qItem{Add packages}{Call pip from the correct env directory. This will install the package in the virtual environment directory instead of the main installation}
			\qItem{Activate}{From \opt{Dir Name} use: source activate.}
			\qItem{Deactivate}{Call: deactivate}
		\end{questionAnswer}
	\end{questions}

\section{MySQL}
	\begin{questions}{Relational Database}
		\rsrc{Wiki}\url{https://en.wikipedia.org/wiki/Relational_database}
		\begin{questionAnswer}
			\qItem{SQL}{Structured Query Language}
			\qItem{Relational Model}{Table rows have unique keys. This allows for columns of 1 table to be linked to columns in another on some shared attribute}
			\qItem{Databases}{\begin{enumerate}
				\item MySQL
				\item Mariadb
				\item sqllite
				\item Postgres
				\item couchdb
			\end{enumerate}}
		\end{questionAnswer}
	\end{questions}

\textbf{Start mysql server: } rcmysql start
