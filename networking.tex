\newcommand{\uls}{Unix \& Linux System Administration Handbook, $4^{th}$}
\chapter{Networking}
\section{Resources}
	\textbf{Books}
		\begin{enumerate}
			\item Beginning Linux Programming (3rd) (See the section on sockets)
			\item Unix Network Programming
			\item Networking for System Administrators
			\item \uls
		\end{enumerate}
	\textbf{Links}
		\begin{enumerate}
			\item \textbf{Network Questions: } \url{https://github.com/kylejohnson/linux-sysadmin-interview-questions/blob/master/test.md}
		\end{enumerate}

\section{Connections}
\begin{questions}{Sockets}
		\FIXME{answer these}

		When a client running a web browser connects to a web server, what is the source port of the connection?

		What is the destination port of the connection?

	\begin{questionAnswer}
		\qItem{Def}{A unix file type with duplex communication}
		\qItem{Use}{Communicating between processes}
		\qItem{List Sockets}{TCP/UDP \ra Socklist, all \ra netstat \& ss}
		\qItem{Listening TCP Sockets}{netstat -tl}
	\end{questionAnswer}

	\begin{itemize}
		\item Attributes: Domain, Type, Protocol
			\begin{itemize}
				\item Domain \ra Address family (UNIX \ra AF\_UNIX, TCP/IP \ra AF\_INET, etc)
				\item Type \ra Communication characteristics
					\begin{itemize}
						\item Stream Sockets (SOCK\_STREAM) \ra Sequenced \& reliable 2 way byte stream. Large messages fragmented, transmitted, \& reassembled. Order of packets is guarenteed
						\item Datagram Sockets (SOCK\_DGRAM) \ra Doesn't establish \& maintain a connection. Unsequenced \& unreliable.
					\end{itemize}
				\item Protocol \ra UNIX and TCP/IP sockets dont require protocols \ra use 0 for the default
			\end{itemize}
		\item Communication Protocols
			\begin{enumerate}
				\item UDP \ra AF\_INET domain with SOCK\_DGRAM connection type
				\item TCP/IP \ra AF\_INET domain with SOCK\_STREAM connection type
				\item Others exist, but are less common
			\end{enumerate}
	\end{itemize}
\end{questions}

\begin{questions}{TCP/IP}
The application determines which communication protocol is more appro-
priate. On the Web, you normally do not want data to go missing during
transmission (a piece of text, image, or downloaded software might get lost,
with annoying to catastrophic results), hence TCP is the correct choice. For television or voice chat, it is usually preferrable to live with small breaks in the service (a pixellated picture or a brief burst of static) than for everything to grind to a halt while the system arranges for a missing datagram to be

	\begin{questionAnswer}
		\qItem{IP Packet}{A data packet sent by the TCP or UDP protocol. Contains header info and data. 20 header bytes and variable number of data bytes}
		\qItem{Local host}{Means \textit{this computer}, connects to the loopback address \ra 127.0.0.1 - 127.255.255.254 (IPv4) \& ::1 (IPv6)}
		\qItem{ARP}{Address resolution protocol. Maps an address (like IPv4 address) to a device (like a MAC address). Same for IPv6 this is done by NDP (see below)}
		\qItem{NDP}{Neighbor Discovery Protocol, removes necessity of DHCP for configuring hosts, although DHCPv6 does exist}
		\qItem{MAC Address}{Media access control address. Unique identifier assigned to network interfaces for communications at the data link layer of a network segment. Also known as Ethernet hardware address (EHA), hardware address or physical address. MAC addresses are supposedly unique world wide. Find current mac w/ arp}
		\qItem{Find an IP or site name}{dig \opt{site name}/\opt{ip address}}
		\qItem{Find site info from DNS}{whois \opt{site name}}
		\qItem{DHCP/DHCPv6}{Dynamic Host Configuration Protocol. Standard network protocol for IP. Dynamically distributes network configuration parameters, such as IP addresses, for interfaces and services}
		\qItem{Default Gateway}{Path to reach all none local connections. Computer \ra Def Gateway (usually a router) \ra ... \ra destinations router \ra destination. Use rout to find gw address}
		\qItem{NAT}{Network address translation. Rerouting IP addresses so that there is only 1 internet routable IP for an entire private network. Used synonomously w/ IP masquarading. Used due to IPv4 exhaustion.}
		\qItem{IPoAC}{IP over Avian Carriers. IP packets carried by pigeon. Mike Tyson IT.}
		\qItem{Subnet mask}{Defines locally reachable connections. etc  192.168.178.0/24 means the first 24 bits are masked away and only the last 8 bits are locally reachable. So 192.168.178.0 to 192.168.178.255 can be reached locally}
		\qItem{CIDR}{Classless Inter-Domain Routing, AKA supernetting \ra removes the necessity for IP classes by masking IP bits by necessity}
		\qItem{Packet filter/firewall}{Filtering based on origin from various IPs}
	\end{questionAnswer}

	IPs (ranges/subnets) reserved for private use/"non-routable" (RFC 1918)?\\
	\begin{tabular}{llll}
		\textbf{IP Class} &\textbf{From} &\textbf{To} &\textbf{CIDR}\\\hline
		Class A& 10.0.0.0& 10.255.255.255& 10.0.0.0/8\\
		Class B& 172.16.0.0& 172.31.255.255& 172.16.0.0/12\\
		Class C& 192.168.0.0& 192.168.255.255& 192.168.0.0/16\\ \hline
	\end{tabular}

How does a switch get a mac address?

What type of packet to discover a router?

A TCP connection on a network can be uniquely defined by 4 things. What are those things?

\end{questions}

\begin{questions}{Internet}
	\begin{questionAnswer}
		\qItem{HTTP/HTTPS}{Hyper text transfer protocol / secure. Request - response protocl for server-client computing. }
		\qItem{SMTP}{Secure messaging transfer protocol}
		\qItem{DNS}{Domain name service, look up IP addresses from human readable names. Use whois or dig as a cmd line tool.}
	\end{questionAnswer}
\end{questions}


\begin{questions}{Tools}
	\begin{questionAnswer}
		\qItem{ifconfig}{Network configuration \& querying the setup of a network interface}
		\qItem{ip}{newer version of ifconfig, use ip addr show to list all connections}
		\qItem{whois}{Look up info in DNS about site}
		\qItem{arp}{Look at the computers hooked up in the subnet and the hardware addresses known}
		\qItem{route}{show / manipulate the IP routing table}
		\qItem{traceroute}{print the route packets trace to network host}
		\qItem{Ping}{Uses the control protocol, ICMP, see if communication is possible. Use ping6 to test IPv6 connections}
		\qItem{LDAP}{Lightweight directory access protocol. A lightweight database for storing various bits of info. Common attributes:\begin{itemize}
			\item \textbf{dn} \ra distinguised name: Search path ex. dn: uid=simon,ou=people,dc=navy,dc=mil
			\item \textbf{o} \ra organization: Often the top level entry
			\item \textbf{ou} \ra organization unit: logical subdivision
			\item \textbf{cn} \ra common name: most natural name to repr entry
			\item \textbf{dc} \ra domain component: used when the model is based on DNS
			\item \textbf{objectClass} \ra Object class: Schema used for this entry
		\end{itemize}}
	\end{questionAnswer}
\end{questions}

\section{Remote Connections}
	\begin{questions}{SSH}
		\begin{questionAnswer}
			\qItem{Encryption}{All communiucations are encryted \ra handshake determines the encryption protocol and proime number, they then share the public keys and keep a secret key}
			\qItem{Keys}{Secret \& public key. Put public key on sever, server sends message to client, client uses secret key to send a return message which confirms the connection.}
			\qItem{Generating keys}{ssh-keygen -t dsa}
			\qItem{X forwarding}{-X (unencrypted), -Y (encrypted)}
			\qItem{File transfer}{SFTP/SCP are the ssh tunnel file transfers, sftp being the upgraded version of scp.}
			\qItem{SSH Hardening}{
				\begin{enumerate}
					\item Disable SSH protocol 1
					\item Reduce the grace time (time to login)
					\item Use TCP wrappers (always good to check)
					\item Increase key strength (maybe go to 2048-bit keys)
					\item Check the defaults and disable a few options
				\end{enumerate}}
		\end{questionAnswer}
	\end{questions}

	\begin{questions}{TLS/SSL}
		\begin{questionAnswer}
			\qItem{TLS}{Transport Layer Security}
			\qItem{SSL}{Secure Sockets Layer}
			\qItem{Encryption}{By key pairing}
			\qItem{Digital certificates}{relies on a set of trusted third-party certificate authorities to establish the authenticity of certificates. Ensures that the public key holder is who they claim to be (perventing man in the middle attacks)}
			\qItem{File transfer}{FTPS \ra FTP SSL or HTTPS \ra HTTP SSL (or secure, etc)}
		\end{questionAnswer}
	\end{questions}

	\begin{questions}{FTP \& Telnet}
		\begin{questionAnswer}
			\qItem{FTP}{File transfer protocol. Often used with SSL liscences for FTPS}
			\qItem{Telnet}{Provides cmd line access to a remote host like ssh. Security concerns has made ssh the prefered communication method}
		\end{questionAnswer}
	\end{questions}

	\begin{questions}{Mail Servers}
		\begin{questionAnswer}
			\qItem{SMTP}{Secure mail transfer protocol}
			\qItem{MX record}{Mail exchange message}
			%\qItem{SMTP sending a message}{\FIXME{}}
		\end{questionAnswer}
	\end{questions}

	\begin{questions}{OSI}
	\FIXME{EXPAND, THIS IS IMPORTANT}, have an example of the different levels and tcp/ip equiv
	Using the OSI model, which layer has the responsibility of making sure that the packet gets where it is supposed to go?
		\begin{questionAnswer}
			\qItem{ISO OSI reference model}{Open Systems Interconnection model. 7 layers each of which only see 1 up and 1 down.}
		\end{questionAnswer}

		\def\totalWidth{40}
		\def\columnA{7}
		\def\columnB{6}
		\def\columnC{17}
		\def\columnD{10}
		\def\exampleColour{paleyellow}
		\begin{center}
			\begin{bytefield}[bitwidth=1.1em]{\totalWidth}
				\bitbox{\totalWidth}{\large\textbf{OSI Model}}\\

				\bytefieldsetup{bitheight=4\baselineskip}%
				\bitbox{\columnA}{\textbf{OSI Layer}} & \bitbox{\columnB}{\textbf{Protocol Data\\Unit (PDU)}} & \bitbox{\columnC}{\textbf{Function}} & \bitbox{\columnD}{\textbf{TCP/IP Example}}\\

				\begin{leftwordgroup}{Host\\Layers}
					\bytefieldsetup{bitheight=5\baselineskip}%
					\colorbitbox{lightgray}{\columnA}{7) Application} & \colorbitbox{lightblue}{\columnB}{Data} & \bitbox{\columnC}{High level APIs including resource sharing and remote file access} & \colorbitbox{\exampleColour}{\columnD}{SSH, FTP, ...}\\

					\bytefieldsetup{bitheight=5\baselineskip}%
					\colorbitbox{lightgray}{\columnA}{6) Presentation} & \colorbitbox{lightblue}{\columnB}{Data} & \bitbox{\columnC}{Translation of data between a networking service and an application; including character encoding, data compression and encryption/decryption} & \colorbitbox{\exampleColour}{\columnD}{Character conversion, data formatting}\\

					\bytefieldsetup{bitheight=5\baselineskip}%
					\colorbitbox{lightgray}{\columnA}{5) Session} & \colorbitbox{lightblue}{\columnB}{Data} & \bitbox{\columnC}{Managing communication sessions, i.e. continuous exchange of information in the form of multiple back-and-forth transmissions between two nodes} & \colorbitbox{\exampleColour}{\columnD}{Network Socket}\\

					\bytefieldsetup{bitheight=5\baselineskip}%
					\colorbitbox{lightgray}{\columnA}{4) Transport} & \colorbitbox{lightblue}{\columnB}{Segment (TCP)\\Datagram (UDP)} & \bitbox{\columnC}{Reliable transmission of data segments between points on a network, including segmentation, acknowledgement and multiplexing} & \colorbitbox{\exampleColour}{\columnD}{TCP or UDP}
				\end{leftwordgroup}\\

				\begin{leftwordgroup}{Media\\Layers}\\
					\bytefieldsetup{bitheight=5\baselineskip}%
					\colorbitbox{lightred}{\columnA}{3) Network} & \colorbitbox{a}{\columnB}{Packet} & \bitbox{\columnC}{ Structuring and managing a multi-node network, including addressing, routing and traffic control} & \colorbitbox{\exampleColour}{\columnD}{IPv4/v6} \\
					\bytefieldsetup{bitheight=5\baselineskip}%
					\colorbitbox{lightred}{\columnA}{2) Data Link} & \colorbitbox{b}{\columnB}{Frame} & \bitbox{\columnC}{Reliable transmission of data frames between two nodes connected by a physical layer} & \colorbitbox{\exampleColour}{\columnD}{Ethernet/wifi adapter} \\
					\bytefieldsetup{bitheight=5\baselineskip}%
					\colorbitbox{lightred}{\columnA}{1) Physical} & \colorbitbox{c}{\columnB}{Bit} & \bitbox{\columnC}{Transmission and reception of raw bit streams over a physical medium} & \colorbitbox{\exampleColour}{\columnD}{Fiber optic, Radio}
				\end{leftwordgroup}
			\end{bytefield}
		\end{center}

	\end{questions}

	\begin{questions}{DNS}
		\rsrc{DNS Record Types} \url{https://en.wikipedia.org/wiki/List_of_DNS_record_types}
		\begin{questionAnswer}
			\qItem{'A' record}{Address record \ra Returns a 32-bit IPv4 address, most commonly used to map hostnames to an IP address of the host, but it is also used for DNSBLs, storing subnet masks in RFC 1101, etc.}
			\qItem{CNAME record}{Canonical name record \ra Alias of one name to another: the DNS lookup will continue by retrying the lookup with the new name.}
			\qItem{'NS' record}{Name server record \ra Delegates a DNS zone to use the given authoritative name servers}
			\qItem{'PTR' record}{Pointer record \ra Pointer to a canonical name. Unlike a CNAME, DNS processing stops and just the name is returned. The most common use is for implementing reverse DNS lookups, but other uses include such things as DNS-SD.}
			\qItem{DNS forwarder}{specific DNS requests are forwarded to a designated DNS server for resolution}
			\qItem{Reverse Lookup}{Double check an IP address by looking up the DN based on the IP}
		\end{questionAnswer}
	\end{questions}

	\begin{questions}{Switches}
		\begin{questionAnswer}
			\qItem{}{}
		\end{questionAnswer}
	\end{questions}

	\begin{questions}{Routing}
		\begin{questionAnswer}
			\qItem{}{}
		\end{questionAnswer}
	\end{questions}

	\begin{questions}{Terms}
		\begin{questionAnswer}
			\qItem{Proxy}{A server that acts as an intermediary for requests from clients seeking resources from other servers.}
			\qItem{IPS}{Internet Provider Security \ra aka registrar tag, used by domain registrar to administer a domain name registration service and related Domain Name System (DNS) services}
			\qItem{DOS}{Denial of service \ra overloading the bandwidth of a server to take it offline}
		\end{questionAnswer}
	\end{questions}
