\documentclass{notes}
\begin{document}

\tableofcontents
\label{sec:front}
\chapter{Linux}
\section{Resources}
	\textbf{Books}
		\begin{enumerate}
			\item Unix and Linux System Administration Handbook (Ordered)
			\item The Practice of System and Network Administration
		\end{enumerate}
	\textbf{Communities}
		\begin{enumerate}
			\item Superuser \ra \url{https://superuser.com/}
			\item Server fault \ra \url{https://serverfault.com/}
			\item Digital Ocean \ra \url{https://www.digitalocean.com/community/tutorials}
		\end{enumerate}
	\textbf{Sites}
		\begin{enumerate}
			\item Ubuntu \ra \url{https://help.ubuntu.com/}
			\item Tutorial Linux \ra \url{https://tutorialinux.com/}
		\end{enumerate}
	\textbf{Links}
		\begin{enumerate}
			\item \url{https://www.slideshare.net/kavyasri790693/linux-admin-interview-questions}
			\item \url{http://simplylinuxfaq.blogspot.in/p/linux-system-admin-interview-questions.html}
			\item \url{https://github.com/kylejohnson/linux-sysadmin-interview-questions/blob/master/test.md}
			\item \url{https://github.com/chassing/linux-sysadmin-interview-questions#hard}
		\end{enumerate}

\newpage

\section{Linux Facts}
	\begin{questions}{Mascots}
		\begin{questionAnswer}
			\qItem{Linux Mascot}{Tux the penguin}
			\qItem{BSD Mascot}{Beastie the Daemon}
		\end{questionAnswer}
	\end{questions}

\section{Users, Passwords \& Permissions}
\begin{questions}{Users}
	\begin{questionAnswer}
		\qItem{Root}{UID : 0, GUID : 0 (root)}
		\qItem{Root Permissions}{RW permissions for all files, but execute privilages can be removed}
		\qItem{pseudo-users}{Have a group w/ special privilages, use su \opt{Group} to login as that group, w/ root this sets the group to the group defined}
		\qItem{Adding a user}{useradd \opt{uname} (single) \ra newusers \opt{batch file} (batch mode useradd). With no args a user is created with the system defaults, usually with a home dir etc.}
		\qItem{Lock an Account}{usermod -l \opt{user}}
		\qItem{New password}{passwd \opt{username}}
		\qItem{Default file permissions}{Set UMASK in /etc/login.defs (debians). Takes away the permissions}
		\qItem{Change Owner \& Group}{chown}
		\qItem{Password \& login info}{/etc/passwd \ra the hashed password itself is held in /etc/shadow}
		\qItem{Change Permissions}{chmod Bit mask OGA rwx}
		\qItem{Delete User}{userdel, removing recusively home folder and files \ra userdel -r}
	\end{questionAnswer}
\end{questions}



\begin{questions}{Groups}
	\begin{questionAnswer}
		\qItem{Wheel}{Group allowing access to the sudo/su command to become another user or the superuser, for sudo this is enabled with visudo.}
		\qItem{Add user to a group}{usermod -a -G \opt{group} \opt{user} (-a only used with -G, without -a, -G makes the given groups the only additional groups he is a member of)}
		\qItem{Change users primary group}{usermod -g \opt{group} \opt{user}}
		\qItem{New Group}{groupadd \opt{group}}
		\qItem{All groups on system}{getent group}
		\qItem{chgrp}{change the group ownership of a file}
	\end{questionAnswer}
\end{questions}

\begin{questions}{Sudo}
	\begin{enumerate}
		\item Add a user as a sudoer by using visudo. You can specify users or groups. Groups have a \% infront to seperate them from users
		\item Common to have a sudo or wheel group and to give that group root permissions in visudo
		\item Syntax \ra \opt{user} computerAddress=(\opt{Runas\_Alias}) \opt{Command\_Alias}
		\item You can use a Runas\_Alias to define a semi-super user that owns a group of files or processes. Then the user can use sudo to run as that user. Same you can limit the commands that a user can run as sudo  with the Command\_Alias
		\item to give sudo root access use: \opt{user} ALL=(ALL) ALL \ra root privilages to \opt{user} with use of sudo
	\end{enumerate}
\end{questions}

\section{Processes}
	\begin{questions}{Process Info}
		\begin{questionAnswer}
			\qItem{PID}{Process ID \ra PID 1 is init, spawns all other ids}
			\qItem{proc}{In /proc \ra State of running processes in a virtual file system}
			\qItem{Process types}{user \ra started w/out special permissions, daemon \ra exist in background, kernel \ra execute only in 'kernel space'}
			\qItem{Forked}{process being started by a parent process}
			\qItem{Nice}{Priority level [-20 (Highest) \ra 19 (Lowest)] \ra 0 is default. Call with: nice \opt{val} \opt{process}, reset the priority level with: renice \opt{new val} \opt{PID}}
			\qItem{Process Monitering}{Top, ps aux, htop \ra good tool}
		\end{questionAnswer}

		\subsubsection{Process states}
			\rsrc{RHEL Doc} \url{https://access.redhat.com/sites/default/files/attachments/processstates_20120831.pdf}

			\begin{questionAnswer}
				\qItem{\textbf{R} \ra Runnable/Running}{
					\begin{enumerate}
						\item Born or forked
						\item Ready to run or runnable
						\item Running in user space or running in kernel space
					\end{enumerate}}
				\qItem{\textbf{S} \ra Sleeping/Waiting}{
					\begin{enumerate}
						\item Present in main memory
						\item Present in secondary memory storage (swap space on disk)
					\end{enumerate}}
				\qItem{\textbf{D} \ra Blocked/Uninterruptable sleep}{Very fast, unobserved, just high priority}
				\qItem{\textbf{T} \ra Temporarily Stopped}{Temporarily stopped but can be restarted}
				\qItem{\textbf{Z} \ra Zombie}{Terminated but parent process has not released it yet}
			\end{questionAnswer}
	\end{questions}

	\begin{questions}{Process Signals}
		\begin{questionAnswer}
			\qItem{kill}{Send a signal to a process with: kill -s \opt{val} (default is 15) \ra see man(7) signals for the signals. Defaults to -9}
			\qItem{pgrep}{Use user or type to find the PID of processes}
			\qItem{pkill}{same as pgrep but it stops the matching PID}
		\end{questionAnswer}
	\end{questions}

\section{Bash Scripting}
\begin{questions}{Shell Variables}
	\begin{questionAnswer}
		\qItem{Set a shell variable from a program output}{\$(arg) or `\opt{arg}`}
		\qItem{getconf}{List system config variables}
		\qItem{export}{Allows a shell variable to be accessed by called processes}
		\qItem{\&\&}{call a command only if the proceeding one exited successfully. \opt{command 1} \&\& \opt{command 2}}
		\qItem{\textbar\textbar}{call a command only if the proceeding one failed. \opt{command 1} \textbar\textbar \opt{command 2}}
	\end{questionAnswer}
\end{questions}

\begin{questions}{Pipes \& Redirection}
	\begin{questionAnswer}
		\qItem{Pipes}{Sends the output of one file into the input of another \ra cat \opt{filename} \textbar~ grep \opt{string}}
		\qItem{Redirect}{Use $>$ to overwrite a file, $>>$ to append. Use 1$>>$ for STDOUT \& 2$>>$ for STDERR, use $>$\& to redirect both.\opt{command} $<$ \opt{file} send the file contents to the command}
	\end{questionAnswer}
\end{questions}

\begin{questions}{General Tools}
	\begin{questionAnswer}
		\qItem{curl}{Tool for talking over several different protocols}
		\qItem{wget}{Downloads files from an address, same as curl but GNU}
		\qItem{uname}{Get kernel \& system information. Use -a for all info.}
		\qItem{df}{Disk free, find the used and available space on the mounted block devices.}
		\qItem{du}{Disk usage, find the space being used by files, scans throught the entire directory passed.}
	\end{questionAnswer}
\end{questions}

\section{Maintenance}
\begin{questions}{Running Jobs}
	\begin{questionAnswer}
		\qItem{Schedule Jobs (user)}{crontab, edit using crontab -e, kept in /var/spool/cron/crontabs, also package specific cron jobs are in /etc/cron.d}
		\qItem{Schedule Jobs (system)}{/etc/crontab}
		\qItem{at}{Run a process at a specified time, accepts HH:MM}
		\qItem{batch}{Run a process when the load drops to a specified level}
		\qItem{Job at boot}{Crontab w/ @reboot}
	\end{questionAnswer}
\end{questions}

\begin{questions}{Backups}
	\rsrc{Backup Tools} \url{http://www.admin-magazine.com/Articles/Using-rsync-for-Backups}

	\rsrc{Rsync Snapshots} \url{http://www.mikerubel.org/computers/rsync_snapshots/}

	\begin{itemize}
		\item rsync \ra Remote/Local, Local/Remote, \& Local/Local file copying. Sends only the differences between the source \& existing files in the destination
			\begin{itemize}
				\item Use: rsync \opt{options} \opt{source} \opt{destination}
					\begin{itemize}
						\item Source \ra Can be files \ra *.c, or everything in a directory \opt{path name}/, remove the trailing slash to copy the directory.
						\item To specify a remote host \opt{computer name} use \ra \opt{computer name}:\opt{path} as the \opt{source} or \opt{destination}. No : means local only.
					\end{itemize}
				\item Options:
					\begin{itemize}
						\item -a \ra Archieve mode, saves symbolic links, devices, attributes, permissions, ownership, groups, and is recursive (i.e. -a == -rlptgoD).
						\item -t \ra Transfer files, if file exists, remote-update protocol is used to update the file by sending only the differences
						\item -z \ra Compress before sending
						\item $--$delete \ra Delete files from the recieving side if not in backup (CAUTION: run $--$dry-run to see what will be removed first)
						\item $--$progress \& -v tells you whats going on
					\end{itemize}
			\end{itemize}
		\item Backup Types \ra All can be done using rsync
			\begin{itemize}
				\item Incrimental \ra Only record changes from last incrimental backup
				\item Differential \ra Records changes since the last total backup
				\item Replica \ra Just replicate the whole shebang
			\end{itemize}
		\item Rsync for incrimental backups
			\begin{itemize}
				\item Have a full backup \opt{Full Backup} \ra rsync to a fresh loc
				\item Have \opt{Backup.0} which has all the incrimental changes
				\item Make each backup look like a full backup using hard links (cp -al)
			\end{itemize}
	\end{itemize}
\end{questions}

\section{Strings \& Searching}
	\begin{questions}{Grep}
		\begin{questionAnswer}
			\qItem{Description}{Search for a character pattern in a string}
			\qItem{Use}{grep \opt{string} \opt{filename} \ra returns the lines with the character pattern \opt{string} in file filename}
			\qItem{\opt{option} -r}{Follow directories}
			\qItem{\opt{option} -n}{Get the line number}
			\qItem{\opt{option} -l}{Get files with the string}
			\qItem{\opt{option} -i}{Ignore case}
		\end{questionAnswer}
	\end{questions}

	\begin{questions}{Strings}
		\begin{questionAnswer}
			\qItem{cut -d \opt{delim} -f \opt{field1}-\opt{field2}}{Break a line on a delim, then take the fields in range, c of chars, b bytes}
			\qItem{sed}{Stream editor, based on the original UNIX tex editor ed.}
			\qItem{awk}{Pattern scanning and processing language}
			\qItem{python}{For normal people use python w/ the re package.}
			\qItem{tr}{Translate, use for replacing certain strings with something else. I mean really just use python, but theoretically use this}
			\qItem{tee}{Put standard in to a file and to standard out, useful for logging the output while filtering}
		\end{questionAnswer}
	\end{questions}

\section{Files}
	\begin{questions}{Files}
		\begin{questionAnswer}
			\qItem{Types}{7 types block special, char special, directory, normal file, symbolic link, named pipe, socket}
			\qItem{diff}{Get difference between 2 files or dirs}
			\qItem{comm}{select or reject common lines between files}
			\qItem{ln -s}{Create a symbolic link \ra sym links dont have to exist unlike hardlinks. Same as a shortcut.}
			\qItem{link/ ln}{Create a hard link \ra file must exist, links and binds the same disk space, if original file is removed, the disk space is still bound to the hard link. Hard links share the same inode.}
			\qItem{Find the file's character set}{file -i \ra gives the mime type, search for binary, ascii etc.}
			\qItem{Inode}{Metadata (information about other data) for files in the file system held in a flat array. Holds ownership, access modes, and file type. Contents:
			\begin{itemize}
				\item Size in bytes
				\item Device ID
				\item User \& Group ID
				\item File Mode (i.e. access)
				\item User Flags
				\item Timestamps \ra file modified, inode modified, and last accessed
				\item Link count
				\item Pointers to disk blocks
			\end{itemize}}
		\end{questionAnswer}
	\end{questions}

	\begin{questions}{File Tools}
		\begin{questionAnswer}
			\qItem{cat}{Read a file}
			\qItem{tac}{Read a file backwards}
			\qItem{Head}{Read first few file lines}
			\qItem{Tail}{Read last few file lines}
			\qItem{read}{read from user input \ra read var \ra will set the var variable}
		\end{questionAnswer}
	\end{questions}

	\begin{questions}{Find}
		\begin{enumerate}
			\item Find a specific file by name find \opt{Starting directory} -name \opt{filename}
			\item Finding by type \ra find \opt{Starting directory} -type \opt{d/f...}
			\item Searching depth \ra find \opt{conditions} -maxdepth \opt{depth}
			\item Running a command on all found files \ra find \opt{conditons} -exec \opt{command} + (the + ends the command, so does $\backslash$; )
			\item Files by last accessed time \ra -atime \opt{days\_ago or -amin min\_ago}
			\begin{enumerate}
				\item a \ra accessed, m \ra modified, c \ra changed
				\item use -daystart to count from the start of the current day instead of right now
				\item use + for greater than the time, - for less and none for exactly
			\end{enumerate}
		\end{enumerate}
	\end{questions}

%\newpage

\begin{questions}{Finding Stuff}
	\begin{questionAnswer}
			\qItem{Locate (mlocate in suse)}{Use updatedb to prepare a database with file locations, then that can be used instead of the slower find}
			\qItem{which}{Shows the full path of (shell) commands (or aliases)}
			\qItem{whereis}{Searches for commands installed and where it is \ra only for programs no aliases}
	\end{questionAnswer}
\end{questions}

\begin{questions}{TAR \& ZIP}
	\begin{questionAnswer}
		\qItem{Make a tarball}{tar -cpf fileout.tar filename1 filename2..., add p to mantain permissions}
		\qItem{Extract a tarball}{tar -xpf filename.tar (be cautious of 'tarbombs' extract in a directory)}
		\qItem{tar \& gzip}{tar -czpf fileout.tar.gz filename1 filename2...}
		\qItem{Uncompress .tar.gz}{tar -xzpf filename.tar.gz}
		\qItem{Compress to .gz}{gzip filename}
		\qItem{Uncompress .gz}{gzip -c filename.gz}
		\qItem{Compress to .Z}{compress filename}
		\qItem{Uncompress .Z}{uncompress filename.Z}
	\end{questionAnswer}
\end{questions}

\section{File System}
	\begin{questions}{Hierarchy (FHS-V2.3)}
		\rsrc{Docs}\url{http://www.pathname.com/fhs/pub/fhs-2.3.pdf}\\
		\begin{questionAnswer}
			\qItem{bin}{Essential command binaries}
			\qItem{boot}{Static files of the boot loader \ra unbootable w/out}
			\qItem{dev}{Device files}
			\qItem{etc}{Host-specific system configuration \ra must be static, cannot be a binary}
			\qItem{home (optional)}{User home dirs}
			\qItem{lib}{Essential shared libraries and kernel modules \ra }
			\qItem{lib$<$qual$>$ (normally lib64 or lib32, optional)}{If multiple library versions are needed like 32 \& 64 bit}
			\qItem{media}{Mount point for removeable media \ra use lsblk to get the names of these}
			\qItem{mnt}{Mount point for mounting a filesystem temporarily}
			\qItem{opt}{Add-on application software packages}
			\qItem{sbin}{Essential system binaries}
			\qItem{srv}{Data for services provided by this system}
			\qItem{tmp}{Temporary files}
			\qItem{usr}{Secondary hierarchy}
			\qItem{var}{Variable data}


			\qItem{root (optional)}{Home dir for root user}
		\end{questionAnswer}
	\end{questions}

	\begin{questions}{Mounting}
		\begin{questionAnswer}
			\qItem{Mounting}{mount /dev/\opt{device} destination}
			\qItem{What disk are mounted}{mount}
			\qItem{Connected disks}{lsblk prints out all of the connected devices nicely formatted}
			\qItem{Mounting on boot}{edit /etc/fstab}
		\end{questionAnswer}
	\end{questions}

	\begin{questions}{RAID}
		\begin{questionAnswer}
			\qItem{Name}{Redundant array of inexpensive/independant disks}
			\qItem{Description}{Combines mutiple storage devices onto one virtualized disk. Used to improve performance and/or reliability}
			\qItem{Performance}{Improves performance by striping data across disks, allowing simultanious read/write operations of multiple disks.}
			\qItem{Reliability}{Mirrors data on multiple disks to deal w/ disk failure.}
			\qItem{Levels}{RAID has has levels 0,1,0+1,1+0,2,3,4,5,\&6}
			\qItem{RAID 0}{Performance \ra stripes data across multiple disks to speed up R/W}
			\qItem{RAID 1}{Reliability \ra aka Mirroring, duplicates data to multiple disks}
			\qItem{RAID 0+1}{ Reliability w/ Performance \ra Mirrors of striped data}
			\qItem{RAID 1+0}{ Performance w/ Reliability \ra Stripped mirrors of data}
			\qItem{RAID 5}{Performance w/ some relibility \ra N-1 disks store data can lose 1 disk}
			\qItem{RAID 6}{Performance w/ Reliability \ra Like RAID 5 but with N-2 disks. Can lose upto 2 disks}
			\qItem{Others}{RAID 2-4 are rarely used.}
			\qItem{JBOD}{Just a bunch of disks (aka linear RAID), combines several disks into a single logical one.}
		\end{questionAnswer}
	\end{questions}

\section{Bootup \& Init}
		\rsrc{yolinux}\href{http://www.yolinux.com/TUTORIALS/LinuxTutorialInitProcess.html}{Init Process}

		\rsrc{tldp}\href{http://www.tldp.org/LDP/intro-linux/html/sect_04_02.html}{Boot, shutdown, init}
	\begin{questions}{Bootup}
		\begin{questionAnswer}
			\qItem{BIOS}{Basic Input/Output System \ra on x86 on boot the computer looks for the bios at the end of the system memory where it is in permemant read-only memory. Looks for bootable media. From the hard disk the bios looks for the MBR.}
			\qItem{MBR}{Master boot record \ra loads the the boot loader (depends on configuration)}
			\qItem{Boot Loaders}{2 common types \ra LILO \& GRUB. Both support multiboot}
			\qItem{Kernel Loading}{Kernel is loaded and has control passed to it}
		\end{questionAnswer}
	\end{questions}

	\begin{questions}{Init}
		\begin{itemize}
			\item Run by the kernel once the kernel is loaded
			\item Parent/ancestor of all automatic processes
			\item The legacy init is sysvinit uses /etc/inittab, systemd is now common and uses . Others include upstart and Linux Standard Base (LSB) init scripts
			\item The first processes that init starts is a script /etc/rc.d/rc.sysinit
		\end{itemize}
		\begin{questionAnswer}
			\qItem{Location}{/sbin/init \ra for systemd this is a sym link to ../usr/lib/systemd/systemd}
			\qItem{PID}{Always PID 1}
			\qItem{Typical Run Levels}{
				\begin{enumerate}\setcounter{enumi}{-1}%start from 0
					\item Halt
					\item Single-user text mode
					\item Unused \ra user definable
					\item Full multi-user text mode
					\item Unused \ra user definable
					\item Full multi-user graphical mode (with an X-based login screen)
					\item Reboot
				\end{enumerate}}
			\qItem{Check runlevel}{use: runlevel}
		\end{questionAnswer}
	\end{questions}

	\begin{questions}{Systemd}
		\begin{itemize}
			\item New default init system
		\end{itemize}
		\begin{questionAnswer}
			\qItem{Boot Scripts}{/etc/systemd/system/ and /lib/systemd/system/}
			\qItem{Service Control}{systemctl \ra Controls the systemd system and service manager}
		\end{questionAnswer}
	\end{questions}


What would you do to recover a lost the root password to a Unix/Linux system?\\
What is a pre-emptive kernel, what does that mean to you?\\
What is the name and location of the system log on a Unix or Linux system?\\
What is the system locale?\\
Where do the login scripts live? Where would I go to find out how many times a user logged in and from where before their account got locked?\\
Where are the DNS, Hostname and most other system wide configuration files? How can you edit them?\\
How could I see if a file system is running out of space. Then: how can you see what is being written that is taking up the most space on that file system? \ra df for space du for usage\\
Why should you never SSH into a production server as root, even if you will be immediately elevating to root?\\
\chapter{Debian}
\section{Debian Resources}
	\textbf{Books}
		\begin{enumerate}
		\item \rsrc{GNU/Linux Desktop Survival Guide}\href{https://www.togaware.com/linux/survivor/}{link}
		\item \rsrc{Debian GNU / Linux 3.1 Bible}\href{http://www.wiley.com/WileyCDA/WileyTitle/productCd-0764595326.html}{link}
		\item\rsrc{The Debian System}\href{http://debiansystem.info/}{link}
		\item\rsrc{The Linux Cookbook 2$^{nd}$}\href{http://dsl.org/cookbook/}{link}
		\item\rsrc{Learning Debian GNU/Linux}\href{http://www.oreilly.com/openbook/debian/book/index.html}{link}
		\item\rsrc{The Debian Administrator's Handbook}\href{https://debian-handbook.info/}{link}
		\end{enumerate}

\newcommand{\uls}{Unix \& Linux System Administration Handbook, $4^{th}$}
\chapter{Networking}
\section{Resources}
	\textbf{Books}
		\begin{enumerate}
			\item Beginning Linux Programming (3rd) (See the section on sockets)
			\item Unix Network Programming
			\item Networking for System Administrators
			\item \uls
		\end{enumerate}
	\textbf{Links}
		\begin{enumerate}
			\item \textbf{Network Questions: } \url{https://github.com/kylejohnson/linux-sysadmin-interview-questions/blob/master/test.md}
		\end{enumerate}

\section{Connections}
\begin{questions}{Sockets}
		\FIXME{answer these}

		When a client running a web browser connects to a web server, what is the source port of the connection?

		What is the destination port of the connection?

	\begin{questionAnswer}
		\qItem{Def}{A unix file type with duplex communication}
		\qItem{Use}{Communicating between processes}
		\qItem{List Sockets}{TCP/UDP \ra Socklist, all \ra netstat \& ss}
		\qItem{Listening TCP Sockets}{netstat -tl}
	\end{questionAnswer}

	\begin{itemize}
		\item Attributes: Domain, Type, Protocol
			\begin{itemize}
				\item Domain \ra Address family (UNIX \ra AF\_UNIX, TCP/IP \ra AF\_INET, etc)
				\item Type \ra Communication characteristics
					\begin{itemize}
						\item Stream Sockets (SOCK\_STREAM) \ra Sequenced \& reliable 2 way byte stream. Large messages fragmented, transmitted, \& reassembled. Order of packets is guarenteed
						\item Datagram Sockets (SOCK\_DGRAM) \ra Doesn't establish \& maintain a connection. Unsequenced \& unreliable.
					\end{itemize}
				\item Protocol \ra UNIX and TCP/IP sockets dont require protocols \ra use 0 for the default
			\end{itemize}
		\item Communication Protocols
			\begin{enumerate}
				\item UDP \ra AF\_INET domain with SOCK\_DGRAM connection type
				\item TCP/IP \ra AF\_INET domain with SOCK\_STREAM connection type
				\item Others exist, but are less common
			\end{enumerate}
	\end{itemize}
\end{questions}

\begin{questions}{TCP/IP}
The application determines which communication protocol is more appro-
priate. On the Web, you normally do not want data to go missing during
transmission (a piece of text, image, or downloaded software might get lost,
with annoying to catastrophic results), hence TCP is the correct choice. For television or voice chat, it is usually preferrable to live with small breaks in the service (a pixellated picture or a brief burst of static) than for everything to grind to a halt while the system arranges for a missing datagram to be

	\begin{questionAnswer}
		\qItem{IP Packet}{A data packet sent by the TCP or UDP protocol. Contains header info and data. 20 header bytes and variable number of data bytes}
		\qItem{Local host}{Means \textit{this computer}, connects to the loopback address \ra 127.0.0.1 - 127.255.255.254 (IPv4) \& ::1 (IPv6)}
		\qItem{ARP}{Address resolution protocol. Maps an address (like IPv4 address) to a device (like a MAC address). Same for IPv6 this is done by NDP (see below)}
		\qItem{NDP}{Neighbor Discovery Protocol, removes necessity of DHCP for configuring hosts, although DHCPv6 does exist}
		\qItem{MAC Address}{Media access control address. Unique identifier assigned to network interfaces for communications at the data link layer of a network segment. Also known as Ethernet hardware address (EHA), hardware address or physical address. MAC addresses are supposedly unique world wide. Find current mac w/ arp}
		\qItem{Find an IP or site name}{dig \opt{site name}/\opt{ip address}}
		\qItem{Find site info from DNS}{whois \opt{site name}}
		\qItem{DHCP/DHCPv6}{Dynamic Host Configuration Protocol. Standard network protocol for IP. Dynamically distributes network configuration parameters, such as IP addresses, for interfaces and services}
		\qItem{Default Gateway}{Path to reach all none local connections. Computer \ra Def Gateway (usually a router) \ra ... \ra destinations router \ra destination. Use rout to find gw address}
		\qItem{NAT}{Network address translation. Rerouting IP addresses so that there is only 1 internet routable IP for an entire private network. Used synonomously w/ IP masquarading. Used due to IPv4 exhaustion.}
		\qItem{IPoAC}{IP over Avian Carriers. IP packets carried by pigeon. Mike Tyson IT.}
		\qItem{Subnet mask}{Defines locally reachable connections. etc  192.168.178.0/24 means the first 24 bits are masked away and only the last 8 bits are locally reachable. So 192.168.178.0 to 192.168.178.255 can be reached locally}
		\qItem{CIDR}{Classless Inter-Domain Routing, AKA supernetting \ra removes the necessity for IP classes by masking IP bits by necessity}
		\qItem{Packet filter/firewall}{Filtering based on origin from various IPs}
	\end{questionAnswer}

	IPs (ranges/subnets) reserved for private use/"non-routable" (RFC 1918)?\\
	\begin{tabular}{llll}
		\textbf{IP Class} &\textbf{From} &\textbf{To} &\textbf{CIDR}\\\hline
		Class A& 10.0.0.0& 10.255.255.255& 10.0.0.0/8\\
		Class B& 172.16.0.0& 172.31.255.255& 172.16.0.0/12\\
		Class C& 192.168.0.0& 192.168.255.255& 192.168.0.0/16\\ \hline
	\end{tabular}

How does a switch get a mac address?

What type of packet to discover a router?

A TCP connection on a network can be uniquely defined by 4 things. What are those things?

\end{questions}

\begin{questions}{Internet}
	\begin{questionAnswer}
		\qItem{HTTP/HTTPS}{Hyper text transfer protocol / secure. Request - response protocl for server-client computing. }
		\qItem{SMTP}{Secure messaging transfer protocol}
		\qItem{DNS}{Domain name service, look up IP addresses from human readable names. Use whois or dig as a cmd line tool.}
	\end{questionAnswer}
\end{questions}


\begin{questions}{Tools}
	\begin{questionAnswer}
		\qItem{ifconfig}{Network configuration \& querying the setup of a network interface}
		\qItem{ip}{newer version of ifconfig, use ip addr show to list all connections}
		\qItem{whois}{Look up info in DNS about site}
		\qItem{arp}{Look at the computers hooked up in the subnet and the hardware addresses known}
		\qItem{route}{show / manipulate the IP routing table}
		\qItem{traceroute}{print the route packets trace to network host}
		\qItem{Ping}{Uses the control protocol, ICMP, see if communication is possible. Use ping6 to test IPv6 connections}
		\qItem{LDAP}{Lightweight directory access protocol. A lightweight database for storing various bits of info. Common attributes:\begin{itemize}
			\item \textbf{dn} \ra distinguised name: Search path ex. dn: uid=simon,ou=people,dc=navy,dc=mil
			\item \textbf{o} \ra organization: Often the top level entry
			\item \textbf{ou} \ra organization unit: logical subdivision
			\item \textbf{cn} \ra common name: most natural name to repr entry
			\item \textbf{dc} \ra domain component: used when the model is based on DNS
			\item \textbf{objectClass} \ra Object class: Schema used for this entry
		\end{itemize}}
	\end{questionAnswer}
\end{questions}

\section{Remote Connections}
	\begin{questions}{SSH}
		\begin{questionAnswer}
			\qItem{Encryption}{All communiucations are encryted \ra handshake determines the encryption protocol and proime number, they then share the public keys and keep a secret key}
			\qItem{Keys}{Secret \& public key. Put public key on sever, server sends message to client, client uses secret key to send a return message which confirms the connection.}
			\qItem{Generating keys}{ssh-keygen -t dsa}
			\qItem{X forwarding}{-X (unencrypted), -Y (encrypted)}
			\qItem{File transfer}{SFTP/SCP are the ssh tunnel file transfers, sftp being the upgraded version of scp.}
			\qItem{SSH Hardening}{
				\begin{enumerate}
					\item Disable SSH protocol 1
					\item Reduce the grace time (time to login)
					\item Use TCP wrappers (always good to check)
					\item Increase key strength (maybe go to 2048-bit keys)
					\item Check the defaults and disable a few options
				\end{enumerate}}
		\end{questionAnswer}
	\end{questions}

	\begin{questions}{TLS/SSL}
		\begin{questionAnswer}
			\qItem{TLS}{Transport Layer Security}
			\qItem{SSL}{Secure Sockets Layer}
			\qItem{Encryption}{By key pairing}
			\qItem{Digital certificates}{relies on a set of trusted third-party certificate authorities to establish the authenticity of certificates. Ensures that the public key holder is who they claim to be (perventing man in the middle attacks)}
			\qItem{File transfer}{FTPS \ra FTP SSL or HTTPS \ra HTTP SSL (or secure, etc)}
		\end{questionAnswer}
	\end{questions}

	\begin{questions}{FTP \& Telnet}
		\begin{questionAnswer}
			\qItem{FTP}{File transfer protocol. Often used with SSL liscences for FTPS}
			\qItem{Telnet}{Provides cmd line access to a remote host like ssh. Security concerns has made ssh the prefered communication method}
		\end{questionAnswer}
	\end{questions}

	\begin{questions}{Mail Servers}
		\begin{questionAnswer}
			\qItem{SMTP}{Secure mail transfer protocol}
			\qItem{MX record}{Mail exchange message}
			\qItem{SMTP sending a message}{\FIXME{}}
		\end{questionAnswer}
	\end{questions}

	\begin{questions}{OSI}
	Using the OSI model, which layer has the responsibility of making sure that the packet gets where it is supposed to go?
		\begin{questionAnswer}
			\qItem{ISO OSI reference model}{Open Systems Interconnection model. 7 layers each of which only see 1 up and 1 down.}
		\end{questionAnswer}
	\end{questions}

	\begin{questions}{DNS}
		\rsrc{DNS Record Types} \url{https://en.wikipedia.org/wiki/List_of_DNS_record_types}
		\begin{questionAnswer}
			\qItem{'A' record}{Address record \ra Returns a 32-bit IPv4 address, most commonly used to map hostnames to an IP address of the host, but it is also used for DNSBLs, storing subnet masks in RFC 1101, etc.}
			\qItem{CNAME record}{Canonical name record \ra Alias of one name to another: the DNS lookup will continue by retrying the lookup with the new name.}
			\qItem{'NS' record}{Name server record \ra Delegates a DNS zone to use the given authoritative name servers}
			\qItem{'PTR' record}{Pointer record \ra Pointer to a canonical name. Unlike a CNAME, DNS processing stops and just the name is returned. The most common use is for implementing reverse DNS lookups, but other uses include such things as DNS-SD.}
			\qItem{DNS forwarder}{specific DNS requests are forwarded to a designated DNS server for resolution}
			\qItem{Reverse Lookup}{Double check an IP address by looking up the DN based on the IP}
		\end{questionAnswer}
	\end{questions}

	\begin{questions}{Terms}
		\begin{questionAnswer}
			\qItem{Proxy}{A server that acts as an intermediary for requests from clients seeking resources from other servers.}
			\qItem{IPS}{Internet Provider Security \ra aka registrar tag, used by domain registrar to administer a domain name registration service and related Domain Name System (DNS) services}
			\qItem{DOS}{Denial of service \ra overloading the bandwidth of a server to take it offline}
		\end{questionAnswer}
	\end{questions}

\chapter{Programing}
\section{GIT}
	\begin{questions}{Setup}
		\begin{questionAnswer}
			\qItem{Get a repo}{git clone}
			\qItem{Make a repo}{git init}
			\qItem{Pull an existing repo}{Use init or clone the repo then pull}
			\qItem{Remote repos}{git remote \ra lists the remote repos, git remote add "name" "url"}
			\qItem{Configuration}{git config \ra complicated, but add email and user with git config --global user.email \& user.name}
		\end{questionAnswer}
	\end{questions}

\section{Terms}
\begin{questions}{Programming}
	\begin{questionAnswer}
		\qItem{Agile}{See below \ref{sect:agile}}
	\end{questionAnswer}
	\begin{enumerate}
		\label{sect:agile}
		\item \textbf{Agile: } Software development strategy. Values:
			\begin{enumerate}
				\item \textbf{Individuals and Interactions} over processes and tools
					\begin{enumerate}
						\item Pair programming \ra 1 station 2 programmers, driver \& navigator/observer
						\item Colocation \ra Team members in the same area
					\end{enumerate}
				\item \textbf{Working Software} over comprehensive documentation
				\item \textbf{Customer Collaboration} over contract negotiation
				\item \textbf{Responding to Change} over following a plan

			\end{enumerate}
	\end{enumerate}
\end{questions}


\end{document}
