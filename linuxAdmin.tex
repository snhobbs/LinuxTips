\documentclass[11pt, noindent]{article}
\usepackage[right=.5in, top=.5in, bottom=1in,left=.5in]{geometry}
\usepackage{hyperref}

\setlength\parindent{0pt}
\setlength\textwidth{7.5in}
\newcommand{\ra}{$\rightarrow$ {}}
\newcommand{\blank}{\_\_\_\_{} }
\newcounter{questionSec}
\newcounter{questionCount}[questionSec]
\newcommand{\qItem}[2]{\stepcounter{questionCount} \textbf{\thequestionCount} & #1 & #2\\}

\newenvironment{questions}[1]{\bgroup
	\centering{\large\textbf{#1}}\\
	\vspace{2pt}\hrule\vspace{12pt}

	\newenvironment{questionAnswer}{
		\begin{tabular}{|r| p{.35\textwidth}|p{.55\textwidth}|}
		\hline\hline
		&&\\
		\stepcounter{questionSec}
	}{\hline\end{tabular}\vspace{11pt}}
}{\egroup}

\begin{document}
\renewcommand\baselinestretch{1.2}

\section{Linux Admin}
\subsection{Resources}
	\textbf{Books}
		\begin{enumerate}
			\item Unix and Linux System Administration Handbook (Ordered)
			\item The Practice of System and Network Administration
		\end{enumerate}
	\textbf{Communities}
		\begin{enumerate}
			\item Superuser \ra \url{https://superuser.com/}
			\item Server fault \ra \url{https://serverfault.com/}
			\item Digital Ocean \ra \url{https://www.digitalocean.com/community/tutorials}
		\end{enumerate}
	\textbf{Sites}
		\begin{enumerate}
			\item Ubuntu \ra \url{https://help.ubuntu.com/}
			\item Tutorial Linux \ra \url{https://tutorialinux.com/}
		\end{enumerate}
	\textbf{Links}
		\begin{enumerate}
			\item \url{https://www.slideshare.net/kavyasri790693/linux-admin-interview-questions}
			\item \url{http://simplylinuxfaq.blogspot.in/p/linux-system-admin-interview-questions.html}
			\item \url{https://github.com/kylejohnson/linux-sysadmin-interview-questions/blob/master/test.md}
			\item \url{https://github.com/chassing/linux-sysadmin-interview-questions#hard}
		\end{enumerate}

\subsection{Users, Passwords \& Permissions}
\begin{questions}{Users}
	\begin{questionAnswer}
		\qItem{Adding a user}{useradd (single) \ra newusers (batch mode useradd)}
		\qItem{Lock an Account}{usermod -l \blank}
		\qItem{New password}{passwd "username"}
		\qItem{Default file permissions}{Set UMASK in /etc/login.defs (debians). Takes away the permissions}
		\qItem{Change Owner \& Group}{chown}
		\qItem{Hashed passwords storage}{/etc/shadow}
		\qItem{Change Permissions}{chmod Bit mask OGA rwx}
		\qItem{Delete User}{userdel, removing recusively home folder and files \ra userdel -r}
	\end{questionAnswer}
\end{questions}

\subsection{Sudo}
	\begin{enumerate}
		\item Add a user as a sudoer by using visudo. You can specify users or groups.
		\item Common to have a sudo or wheel group and to give that group permissions in visudo
		\item Syntax \ra user computerAddress=(Runas\_Alias) Command\_Alias
		\item You can use a Runas\_Alias to define a semi-super user that owns a group of files or processes. Then the user can use sudo to run as that user. Same you can limit the commands that a user can run as sudo  with the Command\_Alias
		\item to give sudo root access use 'user' ALL=(ALL) ALL \ra root privilages to "user" with use of sudo
	\end{enumerate}

\begin{questions}{Groups}
	\begin{questionAnswer}
		\qItem{Wheel}{Group allowing access to the sudo/su command to become another user or the superuser, for sudo this is enabled with visudo.}
		\qItem{Add user to a group}{usermod -a -G "group" "user"(-a only used with -G, without -a, -G makes the given groups the only additional groups he is a member of)}
		\qItem{Change users primary group}{usermod -g "group" "user"}
		\qItem{New Group}{groupadd \blank}
		\qItem{All groups on system}{getent group}
		\qItem{chgrp}{change the group ownership of a file}
	\end{questionAnswer}
\end{questions}

\begin{questions}{Mounting}
	\begin{questionAnswer}
		\qItem{Mounting}{mount /dev/\blank destination}
		\qItem{What disk are mounted}{mount}
		\qItem{Connected disks}{lsblk prints out all of the connected devices nicely formatted}
		\qItem{Mounting on boot}{edit /etc/fstab}
	\end{questionAnswer}
\end{questions}

\begin{questions}{TAR \& ZIP}
	\begin{questionAnswer}
		\qItem{Make a tarball}{tar -cf fileout.tar filename1 filename2...}
		\qItem{Extract a tarball}{tar -xf filename.tar (be cautious of 'tarbombs' extract in a directory)}
		\qItem{tar \& gzip}{tar -czf fileout.tar.gz filename1 filename2...}
		\qItem{Uncompress .tar.gz}{tar -xzf filename.tar.gz}
		\qItem{Compress to .gz}{gzip filename}
		\qItem{Uncompress .gz}{gzip -c filename.gz}
		\qItem{Compress to .Z}{compress filename}
		\qItem{Uncompress .Z}{uncompress filename.Z}
	\end{questionAnswer}
\end{questions}

\begin{questions}{Files}
	\begin{questionAnswer}
		\qItem{Types}{7 types block special, char spectial, directory, normal file, symbolic link, named pipe, socket}
		\qItem{diff}{Get difference between 2 files or dirs}
		\qItem{comm}{select or reject common lines between files}
		\qItem{ln}{Create a symbolic link}
		\qItem{link}{Create a hard link}
	\end{questionAnswer}
\end{questions}

\begin{questions}{Shell Variables}
	\begin{questionAnswer}
		\qItem{Set a shell variable from a program output}{\$(arg)}
		\qItem{getconf}{List system config variables}
	\end{questionAnswer}
\end{questions}

\begin{questions}{Pipes \& Redirection}
	\begin{questionAnswer}
		\qItem{Pipes}{Sends the output of one file into the input of another \ra cat \blank | grep "\blank"}
		\qItem{Redirect}{Use $>$ to overwrite a file, $>>$ to append. Use 1$>>$ for STDOUT \& 2$>>$ for STDERR}
	\end{questionAnswer}
\end{questions}

\begin{questions}{General Bash}
	\begin{questionAnswer}
		\qItem{curl}{Tool for talking over several different protocols}
	\end{questionAnswer}
\end{questions}

\begin{questions}{Maintenance}
	\begin{questionAnswer}
		\qItem{Schedule Jobs (user)}{crontab, edit using crontab -e, kept in /var/spool/cron/crontabs, also package specific cron jobs are in /etc/cron.d}
		\qItem{Schedule Jobs (system)}{/etc/crontab}
		\qItem{at}{Run a process at a specified time, accepts HH:MM}
		\qItem{batch}{Run a process when the load drops to a specified level}
	\end{questionAnswer}
\end{questions}

\subsection{Strings \& Searching}
	\begin{questions}{Bash Strings}
		\begin{questionAnswer}
			\qItem{cat}{Read a file}
			\qItem{tac}{Read a file backwards}
			\qItem{nl}{Number lines in output}
			\qItem{Head}{Read first few file lines}
			\qItem{Tail}{Read last few file lines}
			\qItem{read}{read from user input \ra read var \ra will set the var variable}
			\qItem{cut}{Break a line on a delimiter}
		\end{questionAnswer}
	\end{questions}

	\subsubsection{Grep}
		\begin{enumerate}
			\item Search for a character pattern in a string
			\item grep \blank filename \ra returns the lines with the character pattern \blank in file filename
			\item Follow directories "grep -r \blank ./*"
			\item Get the line number \ra -n
			\item Get files with the string \ra -l
			\item Ignore case \ra -i
		\end{enumerate}

	\subsubsection{Find}
		\begin{enumerate}
			\item Find a specific file by name find \{Starting directory\} -name "filename"
			\item Finding by type \ra find \{Starting directory\} -type d/f...
			\item Searching depth \ra find \blank -maxdepth "depth"
			\item Running a command on all found files \ra find \blank \blank -exec "command" {} + (the + ends the command)
			\item Files by last accessed time \ra -atime "days\_ago" or -amin "min\_ago".
			\begin{enumerate}
				\item a \ra accessed, m \ra modified, c \ra changed
				\item use -daystart to count from the start of the current day instead of right now
				\item use + for greater than the time, - for less and none for exactly
			\end{enumerate}
		\end{enumerate}

	\begin{questions}{file}
		\begin{questionAnswer}
			\qItem{Find the file's character set}{file -i \ra gives the mime type, search for}
			\qItem{tac}{Read a file backwards}
			\qItem{Head}{Read first few file lines}
			\qItem{Tail}{Read last few file lines}
			\qItem{read}{read from user input \ra read var \ra will set the var variable}
			\qItem{cut -d : -f "field1"-"field2"}{Break a line on a delim ':', then take the fields in range, c of chars, b bytes}
		\end{questionAnswer}
	\end{questions}


\section{GIT}
	\begin{questions}{Setup}
		\begin{questionAnswer}
			\qItem{Get a repo}{git clone}
			\qItem{Make a repo}{git init}
			\qItem{Pull an existing repo}{Use init or clone the repo then pull}
			\qItem{Remote repos}{git remote \ra lists the remote repos, git remote add "name" "url"}
			\qItem{Configuration}{git config \ra complicated, but add email and user with git config --global user.email \& user.name}
		\end{questionAnswer}
	\end{questions}

\section{MySQL}
\subsection{Users \& Permissions}

\section{Python}


\section{Networking}
	\textbf{Links: }\url{https://github.com/kylejohnson/linux-sysadmin-interview-questions/blob/master/test.md}

\section{Terms}
\begin{questions}{Programming}
	\begin{questionAnswer}
		\qItem{Agile}{See below \ref{sect:agile}}
	\end{questionAnswer}
	\begin{enumerate}
		\label{sect:agile}
		\item \textbf{Agile: } Software development strategy. Values:
			\begin{enumerate}
				\item \textbf{Individuals and Interactions} over processes and tools
					\begin{enumerate}
						\item Pair programming \ra 1 station 2 programmers, driver \& navigator/observer
						\item Colocation \ra Team members in the same area
					\end{enumerate}
				\item \textbf{Working Software} over comprehensive documentation
				\item \textbf{Customer Collaboration} over contract negotiation
				\item \textbf{Responding to Change} over following a plan

			\end{enumerate}
	\end{enumerate}
\end{questions}
\end{document}
